\section{Các công trình nghiên cứu liên quan}

Trong lĩnh vực khuyến nghị khóa học, nhiều công trình nghiên cứu đã được thực hiện nhằm cải thiện hiệu quả và độ chính xác của các hệ thống khuyến nghị. Một số phương pháp nổi bật được đề xuất như sau:

\subsection{Matrix Factorization}

Matrix Factorization (MF) là một kỹ thuật phổ biến trong hệ thống khuyến nghị, được sử dụng để phân rã ma trận user-item nhằm phát hiện các yếu tố tiềm ẩn ảnh hưởng đến hành vi của người dùng. Koren và cộng sự (2009) đã giới thiệu phương pháp này và áp dụng vào hệ thống khuyến nghị, đặc biệt trong bài toán đề xuất phim. Phương pháp này cho phép mô hình hóa sự tương tác giữa người dùng và các khóa học dựa trên các đặc trưng tiềm ẩn, mang lại kết quả tốt trong nhiều bài toán thực tế.

\subsection{Collaborative Filtering}

Collaborative Filtering (CF) là một trong những phương pháp lâu đời nhất và hiệu quả nhất trong khuyến nghị. CF được chia thành hai nhánh chính:
\begin{itemize}
    \item \textbf{User-User Collaborative Filtering}: Dựa trên sự tương đồng giữa các người dùng để gợi ý các khóa học mà người dùng có thể quan tâm.
    \item \textbf{Item-Item Collaborative Filtering}: Dựa trên sự tương đồng giữa các khóa học để gợi ý các khóa học mới cho người dùng.
\end{itemize}
Su và Khoshgoftaar (2009) đã thực hiện một khảo sát toàn diện về CF và các phương pháp cải tiến nhằm khắc phục những hạn chế của nó, bao gồm vấn đề thưa thớt dữ liệu và mở rộng quy mô.

\subsection{Content-Based Filtering}

Content-Based Filtering (CBF) là phương pháp tập trung vào các đặc trưng của khóa học, dựa trên thông tin về nội dung của các khóa học mà người dùng đã tham gia để đưa ra gợi ý. Burke (2002) đã tổng hợp các phương pháp CBF và chỉ ra rằng việc kết hợp các đặc trưng nội dung của khóa học và hành vi người dùng có thể nâng cao hiệu quả gợi ý.

\subsection{Graph-Based Recommender Systems}

Phương pháp dựa trên đồ thị (Graph-Based Recommender Systems) sử dụng cấu trúc đồ thị để biểu diễn mối quan hệ giữa người dùng và khóa học. Các thuật toán như Random Walk hoặc PageRank được áp dụng để tìm kiếm và khai thác các kết nối trong dữ liệu. Phương pháp này đặc biệt hiệu quả khi xử lý dữ liệu phức tạp với nhiều mối quan hệ đa chiều.

\subsection{Các hệ thống khuyến nghị khóa học dựa trên dữ liệu lớn}

Trong bối cảnh dữ liệu lớn, các nền tảng như MOOCCubeX cung cấp một lượng dữ liệu khổng lồ về hành vi học tập của người dùng. Zhang và cộng sự (2022) đã phát triển một hệ thống khuyến nghị dựa trên dữ liệu từ MOOCCubeX, sử dụng các mô hình đồ thị để cá nhân hóa lộ trình học tập. Kết quả cho thấy hệ thống có khả năng gợi ý các khóa học phù hợp và tăng tỷ lệ hoàn thành khóa học.

\subsection{Ứng dụng của các mô hình học sâu}

Học sâu (Deep Learning) được áp dụng rộng rãi trong các hệ thống khuyến nghị khóa học hiện đại. Các mạng nơ-ron tích chập (Convolutional Neural Networks - CNNs) và mạng nơ-ron hồi tiếp (Recurrent Neural Networks - RNNs) được sử dụng để phân tích các chuỗi hành vi học tập của người dùng. Ngoài ra, Transformer và mô hình Attention cũng được triển khai để tối ưu hóa việc gợi ý dựa trên các đặc trưng tuần tự và ngữ cảnh.

\subsection{Tổng quan các công trình nghiên cứu}

Tóm lại, các phương pháp khuyến nghị khóa học đã có nhiều bước tiến đáng kể nhờ vào việc kết hợp các kỹ thuật truyền thống và hiện đại, từ Matrix Factorization đến Neural Collaborative Filtering. Các nghiên cứu không chỉ tập trung vào việc cải thiện độ chính xác mà còn nhấn mạnh đến khả năng mở rộng, xử lý dữ liệu lớn, và cá nhân hóa trải nghiệm học tập cho từng người dùng.
